\documentclass[twocolumn,10pt]{article}
\usepackage{epsf,verbatim} %,mystyle}

\setlength{\oddsidemargin}{-0.1in}	% 1 (real) inch for right margin
\setlength{\evensidemargin}{-0.1in}	% 1 (real) inch for left margin
\setlength{\textwidth}{6.5in}
\setlength{\textheight}{8.5in}
\setlength{\topmargin}{-0.45in}
\renewcommand{\textfraction}{.4}  % fraction of page for text
\renewcommand{\floatpagefraction}{.8} % fraction of float page to be no text


\title{Navigating in Hyperspace: A Structure Based Toolbox}

\author{Ehud Rivlin\thanks{Department of Computer Science \&
Human-Computer Interaction Laboratory,
University of Maryland,
College Park, MD 20742-3411}\ \thanks{e-mail: ehud@cs.umd.edu}
\and 
Rodrigo Botafogo\thanks{C\&C Information Technology Research
Laboratories, NEC Corporation 4-1-1, Miyazaki, Miyamae-Ku, Kawasaki,
Kanagawa 216, Japan, botafogo\%joke.cl.nec.co.jp@uunet.uu.net} 
\and 
Ben Shneiderman$^*$\thanks{Send correspondence to Ben Shneiderman,
ben@cs.umd.edu}}

\date{}

\begin{document}
\maketitle


\begin{abstract}

Analyzing the structure of a hypertext database can give useful
information to the traveler in hyperspace. We present a preliminary
collection of structural tools for users of hypertext systems. These
tools can suggest answers to questions like: Where am I? How can I
choose and get to my destination? What else is in my current
neighborhood? etc.  Structure is imposed on the hypertext by using two
processes: hierarchisation and cluster identification.  Several
metrics are presented and used in the above processes for locating
landmarks and getting global information on the hypertext structure.
The structural analysis is integrated with previous attempts to reduce
the users' disorientation while navigating the hyperspace. An
integration with fisheye views and tree-maps is presented.

\end{abstract}

\section{Introduction}
\paragraph{}

As more books and other documents become available in electronic form,
the use of hypertext 
systems is becoming more common. But as the size of hypertext databases
grows the ``lost in hyperspace'' problem may limit
efficient and meaningful usage of hypertext systems~\cite{nie90}.
Good solutions to the problem need to be holistic ~\cite{nor}; encompassing
navigation, planning, annotation etc.

We concentrate on the navigational aspects of the problem from a
structural point of view and propose a set of tools that can help
hypertext users reduce the cognitive load of navigation. 
Nielsen~\cite{nie90} describes the user's disorientation while
navigating as ``one of the major usability problems with hypertext''.
Our tools can suggest answers to questions like: Where am
I?\footnote{In~\cite{nie90} Nielsen quotes previous work where 56
precent of the readers agreed (fully or partly) with the statment {\em
I was often confused about ``where I was''}.} How do I get to any
destination? What can be seen from where I am? etc. These tools are
based on a structural analysis of the hypertext. The analysis, and the
answers we get, reflect the input that we are using - the structure.
We can not get more than what a structural analysis can give. The
results can be strengthened if the analysis were extended to include
textual and stylistic dimensions as well. This natural extension is
beyond the scope of this paper. The results presented can be
integrated nicely with previous solutions to the problem, like the
ones that are based on improved user interfaces, multiple
windows~\cite{mar89}; or maps~\cite{nie90}.

In this paper the word ``hypertext'' is used to indicate 
hypertext systems where nodes (cards, articles, documents, ...)
and links are untyped, and where the hypertext
can be represented by a directed graph\footnote{Most of the hypertext systems have a backup function that
gives the user the ability to go back to the last node that was visited.
This option changes the hypertext from a directed to undirected graph.
Although this fact is important from a user interface point of view, 
an analysis of the hypertext as a directed graph
(without taking into account the backup option) reveals some properties
of the structure.}.
This paper describes results working with 3 different hypertexts
created in Hyperties\footnote{Hyperties is a hypertext system developed by the Human 
Computer Interaction Laboratory, now distributed and improved by 
Cognetics Corporation, Princeton
Junction, NJ.}.

Section 2 introduces some concepts and metrics used for our
analysis\footnote{Most of them were introduced in~\cite{bot90}.}.
Section 3 presents some possible answers to the question: Where am
I? The answers are based on the ability to find hierarchies in 
the hypertext, as well as the ability to create aggregates based
on graph theoretic algorithms. 
Section 4 describes the use of different metrics in giving 
users some notion of the hyperspace structure. For this
purpose we also show how to integrate the structural analysis
with previous work such as tree-maps and fisheye views.
Section 5 points out the next move, namely, how to plan a path.
 


\section{Concepts and metrics}
\subsection*{The converted distance matrix}
\paragraph{}
A matrix that has as its entries the distances of every node ($n$ nodes
in the hypertext) 
to every other
node is called the {\em distance matrix} of a graph. Let M be the distance
matrix of the graph.
The {\em converted distance matrix} is defined as follows:
\[ C_{ij} = \left\{ \begin{array}{ll}
                      M_{ij} & \mbox{$if M_{ij} \neq \infty$,}     \\
                      K      & \mbox{otherwise}
                    \end{array}
            \right.
\]
Where K is a finite constant (the conversion constant). Typically K = n.

\subsection*{The converted and the relative out and in centrality metrics}
\paragraph{}
The converted out distance (COD) for a node is the sum of all entries
in row {\em i} in the converted distance matrix (C)
\[ COD_{i} = \sum_{j} C_{ij} \]
The converted distance (CD) of a hypertext is given by the sum of all
entries in the converted distance matrix
\[ CD = \sum_{i} \sum_{j} C_{ij} \]
The COD is a good indication of the centrality of a node within a given
hypertext. The need for normalization leads to the definition of
the {\em relative out centrality} (ROC)
\[ ROC_{i} = CD / COD_{i} \]

A high ROC indicates a node that can easily access other
nodes (high out centrality). Hence, the higher the ROC of a node 
the more central it is.
In a similar manner we can define the converted in distance (CID) for a node
{\em i} as the sum of all entries in the column {\em i} in the converted 
distance matrix (C)
\[ CID_{i} = \sum_{j} C_{ji} \]
The {\em relative in centrality} (RIC) is defined as:
\[ RIC_{i} = CD / CID_{i} \]
A high RIC indicates a node that is easily accessible (high in centrality).

\begin{figure}
\vspace*{8cm}
\caption [10pt] {Graph with its converted distance matrix and the 
associated metrics.}
\label{fig:roc}
\end{figure}

Figure~\ref{fig:roc} shows a graph with the associated converted 
distance matrix and
its COD, ROC, CID, RIC. The value of K was chosen to be the number of nodes
in the hypertext\footnote{The higher the value of K, the more central is the
requirement that a node needs to reach {\bf all} the other nodes in order to
have high ROC. See~\cite{bot90}.}.
Node $c$ has the highest ROC and intuitively we see that this is 
the most central node in the graph. 

\section{Where am I? Revealing the structure}
\paragraph{}
A possible way to localize a user in a hypertext is to impose a
structure on the hypertext, and to identify the user's location within
that structure. A natural structure is a hierarchy, and some systems
like KMS require a hierarchical structure~\cite{kms}. The role of
hierarchy formation in reading is presented in~\cite{cha87,van83}. 
In~\cite{bot90} we proposed an algorithm for hierarchisation of
hypertexts. The hierarchisation process is a two phase algorithm: first
a root is identified and then hierarchical and cross-reference links
are distinguished.

A good candidate for a root is a node with a high ROC. Intuitively this is a 
node that can access a large number of nodes and the distance from it to 
other nodes is small. In order for the algorithm to give meaningful results
a ``noise removal'' process should precede the ROC computation. In this 
process we remove index nodes, i.e. nodes that points to a very large portion
of the nodes in the hypertext\footnote{If we define $\alpha$ to be the mean of the outdegrees of the nodes
in the hypertext, an index node is defined to be a node whose outdegree is 
greater than $\alpha$ plus some predefined threshold. In order to create a
``cleaner'' structure we can take out reference nodes as well. Reference
nodes are in a certain way the converse of index nodes, and are defined 
similarly on the indegree, see~\cite{bot90}.}.

\begin{figure}
\vspace*{8cm}
\caption [10pt] { Hierarchisation of a graph.}
\label{fig:hier}
\end{figure}

The differentiation between hierarchical and cross-reference links is done
with a variation of breadth first search trying to maintain nodes as close to
the root as possible. Figure~\ref{fig:hier} shows a graph and its 
hierarchisation using 
node $c$ as the root. Hierarchical links are represented with normal lines, 
while cross-reference links are represented with dashed-lines. Note that a 
node which has two parents from the same level appears twice, i.e. node $f$.
In this case redundant subtrees were created. This situation raises the option
for two semantic interpretations for node $f$: one as part of the subtree
headed by node $a$, the other as part of the subtree headed by $e$.

Once a hierarchy has been found it is important to show it to the user
in a reasonable way. One option is to display the entire generated
tree.  This option might not be practical for a hypertext with a large
number of nodes (this is of course a function of the linearity of the
tree as well).  Another option might be a fisheye based~\cite{fur86}
presentation.  This presentation is based on the observation that
humans usually represent their own neighborhood with great detail, yet
only major landmarks further away. The importance of a node (the a
priori interest) can be taken to be the distance from the root. This
option is discussed in the next section.

\begin{figure}
\vspace*{7cm} 
\caption [10pt] { Asking: Where am I? being at the node ``CMSC 412''. 
Hierarchisation of the CMSC hypertext. Only 
clusters on the path from the root are presented.}
\label{fig:cmsc412}
\end{figure}

We applied the hierarchisation algorithm to the CMSC hypertext.  CMSC
is a hypertext with 106 nodes and 402 links which describes the
computer science department at the University of Maryland. The
hypertext contains a description of the facilities at the department,
biographies of the faculty members, details about the undergraduate
and the graduate programs as well as a description of the courses
given at the department.  In running the algorithm we identified three
index nodes (using as a threshold three times the standard deviation,
which gave 18 nodes above the threshold) and removed them.  The node
having highest ROC was the ``Introduction'' (ROC = 1838).  The second
highest ROC was 1000 and from there on the ROC droped significantly,
with the 10{\em th} highest ROC equal to 400.  This reflects the fact
that CMSC is a well structured hypertext, with a clear node as the
root. Some other hypertexts that we looked at did not have this
feature. GOVA\footnote{GOVA stands for Guide to Opportunities in
Volunteer Archaeology. It is a museum application by the Smithsonian. 
The hypertext contains 222 nodes and 1609 links and is highly
interconnected.  In this application users would use the hypertext to
find places around the world where they are interested in becoming
volunteer archaeologist. Using a touchscreen they were be able to
select in maps a region of the world or even a particular city in
which they were interested. Once the place was selected the user would
then read a text explaining the price, best time to go, etc. The
system was not limited to only those operations, users had at their
fingertips information about the sites: period of history, geographic
information and others that were all linked forming a network of
information.}~\cite{pla} for example, is a hypertext which contains
information for users who want to become volunteer archaeologists. It
contains information about sites, archaeology in general,
accommodations and so on. The hypertext is a collaborative work of
several writers, who did not have a consistent point of view. This
resulted in small differences between the candidates for the root
without any dominant node.  In CMSC, on the other hand, there is a
coherent structure aimed to present the department. This fact is
reflected in a single clear candidate for the root.

Figure~\ref{fig:cmsc412} shows the result of asking: ``Where am I?''
being at the node ``CMSC 412 - Operating Systems''. This node
describes the advanced undergraduate course (or low level graduate) in
operating systems. Choosing the node ``Introduction'' to be the root,
the algorithm found this node as one that belongs to the ``Graduate
Courses'' cluster. This is one of the eight nodes directly connected
to the root.  The clusters varied from laboratory facilities to fields
of research, University of Maryland, graduate courses etc.  The
``Graduate Courses'' cluster (or chapter) has 6 subclusters, each
representing a different group of courses from a different domain.
The node ``CMSC 412 - Operating Systems'' was under the subcluster
``Computer Systems Courses''. Thus, the user will get the following
answer: You are in the subcluster of ``Computer System Courses'', that
is a subcluster of the ``Graduate Courses'' cluster which is connected
to the root (``Introduction'').

An ambiguous answer is possible when querying: ``Where am I?'' being
in the node ``Rosenfeld, Azriel''. According to the algorithm the node
is included in two subclusters\footnote{This is exactly the case as
with node $f$ in figure 2} , namely ``Research in Computer Vision''
and ``Center for Automation Research'', which in turn are connected
respectively to the clusters ``Fields of Research'' and
``Collaboration with Industry''.  This situation is natural. Prof. 
Rosenfeld is working in the area of computer vision which is one of
the fields of research in the computer science department. On the
other hand Prof.  Rosenfeld is the head of the Center for Automation
Research, which has connection to industry. These two views are
equally justified, and both will be presented to the user
(figure~\ref{fig:rosenfeld}) This kind of ambiguity can be solved by
using the user's history as a guide.  This, of course, can be the
user's choice.


\begin{figure}
\vspace*{8cm}
\caption [10pt] { Asking: Where am I? being at the node ``Rosenfeld Azriel''.
Two views that are equally justified.}
\label{fig:rosenfeld}
\end{figure}

For comparison we present an overview map of this node
(figure~\ref{fig:overview}) The map presents all the neighbors of this
node at distance one, and most of the neighbors at distance two.  Even
in a relatively small scale hypertext the number of the possible links
in a small environment\footnote{Looking at a node and its immediate
neighborhood, defined in~\cite{nie90} as local map} can be quite high
(and confusing).

\begin{figure}
\vspace*{8cm}
\caption [10pt] { The node ``Rosenfeld Azriel'', its distance-1
neighborhood and some nodes at distance 2.}
\label{fig:overview}
\end{figure}

It is interesting to note that picking different nodes as a root gives
a different view of the same hypertext. For example, running the
algorithm on GOVA resulted in ten candidates for a root.  Different
perspectives resulted from taking a different root.  For example
taking ``Archaeological projects'' to be the root we find the node
``Caesarea's harbor'' under the cluster ``Harbor archaeology'' which
is connected to the root. When taking ``Introduction'' to be the root
we find the node ``Caesarea's harbor'' under the cluster ``Caesarea''
which is connected to the root. This view results from the Smithsonian
presentation being centered around excavation in Caesarea and being
called ``King Herod's Dream''. Viewing the hypertext from the
``Introduction'', Caesarea was a cluster, a central part in the
presentation. Viewing from ``Archaeological projects'' Caesarea was
only another site and that is why it was found on level 5 (and of
course wasn't a main cluster).

\begin{figure}
\vspace*{8cm}
\caption [10pt] {Two different answers being at the 
node ``Caesara's harbor''.}
\label{fig:caesara}
\end{figure}

\subsection*{Using aggregation}
\paragraph{}
Aggregation is an operation that clusters related objects and forms a
higher level object. Halasz~\cite{hal88} identifies aggregation as a
composition mechanism. A composition mechanism is a way of dealing with
a set of nodes and links as a single object. In a hypertext database the
presence of a link between two articles indicates that there is a 
semantic relation between those two articles. The semantic relationship
between the nodes is reflected in the structure.   
 
In~\cite{bot91} two graph theoretic algorithms that identify semantic
clusters were presented\footnote{A semantic cluster of a hypertext is a set of nodes and links
that are a subgraph of the hypertext graph, and the compactness of the 
subgraph is higher than the compactness (defined in section 4) 
of the whole graph. This is
a set of nodes appropriate for aggregation. See~\cite{bot91}.}
. The algorithms are based on finding the
biconnected components\footnote{A graph G is biconnected if it is connected and has no 
articulation points. A node $a$ of a connected graph is an 
articulation point if the subgraph obtained by deleting $a$ and
all the edges incident on $a$ is no longer connected. As a result
biconnected components in a graph have the property that there are
at least two paths between any two nodes in the component.}
and strongly connected components\footnote{The same definition but for directed graphs.} 
of a graph. First outgoing edges from index nodes and incoming
edges to reference nodes are removed, and then a reduced graph is built 
by breaking the graph into its biconnected components. The algorithms are
recursive in the sense that every bicomponent found is treated as an 
independent graph. These algorithms give users the option to break the 
hypertext
into semantic clusters. Users can display the reduced graph and see 
their location.

In figure~\ref{fig:reducedg} part of the reduced graph of CMSC is
presented, after the first run of the biconnected components algorithm.
The algorithm divided the graph into a central core and some subparts.
A similar partition (central, relatively large, core) resulted when the
algorithm was run on the HHO hypertext\footnote{``HyperText Hands-On!'' is the first
electronic book commercially published. The hypertext covers
concepts of hypertext, 
typical hypertext applications, and currently available authoring
systems. The hypertext also covers design and implementation issues such
as user interface, performance and networks. The hypertext has 243 nodes
and 803 links.}
(see figure~\ref{fig:hho}). Users
being, for example, in the node ``Howard, Elman'' can see some
different clusters that constitute the hypertext, and  their semantic
meaning (a pleasing outcome that can not be always guaranteed). 
A further application of the process to
the cluster that contains the node ``Howard, Elman'' gives more
information. Users can see that the nodes ``O'Leary, D.'' and
``Stewart, G. W.'' create an appropriate subcluster
because these three faculty members constitute the numerical analysis
group. This is an interesting piece of semantic information
that is given by the structure. The breaking of the hypertext
into components, and being able to see their connections,
 is important, and might give users a notion
of their position by using the different clusters as landmarks.

\begin{figure}
\vspace*{9cm}
\caption [10pt] { Part of the reduced graph of CMSC. The user is in the node
 ``Howard, Elman'', in the ``Research in numerical analysis'' cluster.
The cluster is a reduced graph that contains two nodes and a subcluster.}
\label{fig:reducedg}
\end{figure}

\section{Where am I? Looking around}
\paragraph{}
The previous section presented tools to help users toq localize themselves 
in the hyperspace. This section presents tools to give 
users useful information about their environment.
The information is gained by defining and analysing metrics on the
hypertext. We use global as well as local metrics.
Global metrics are concerned with the
hypertext as a whole, while node, or local, metrics focus on properties
of individual nodes.

\subsection*{Local metrics}
\paragraph{}
A simple metric that can give users a useful hint about their position
is depth. By definition, the depth of a node in a hypertext is its distance
from the root. The hierarchisation algorithm imposes a tree structure
on the hypertext and by doing so gives the depth information.

Two inter-related local metrics are the {\em status} and {\em
contrastatus} (first defined in~\cite{har59}). The status is the sum
of the finite entries in the $i^{th}$ row of the distance matrix, and
the contrastatus is the sum of the finite entries in the $j^{th}$
column of the distance matrix.  When a node has a very low status, for
example, it means that it cannot reach many of the other nodes in the
hypertext. When a node has very low contrastatus it means that it
cannot be reached by many nodes.  The {\em prestige} of a node is
define to be the difference between its status and contrastatus. Note
that the prestige might have negative values too.
Figure~\ref{fig:prest} presents the graph from figure~\ref{fig:roc}
with the values for the different metrics. Nodes that reach all the
other nodes in the hypertext have status identical to their COD
metric.  There is a difference between the ROC and the status (and the
prestige).  Node $c$, for example, has the highest ROC value, but node
$e$ is more ``prestigious'' (its subordinates are further away).

\begin{figure}
\vspace*{8cm}
\caption [10pt] { The graph from figure 1 with the status, contrastatus,
and the prestige.}
\label{fig:prest}
\end{figure}

The local metrics can be used as guidance for users. Readers might
find out that after wandering about they are located only two nodes
away from the root (depth 3). Being in a node with low status and high
contrastatus might suggest the use of aggregation instead of
hierarchisation as in the following example.  In ``Hypertext
Hands-On!'' (HHO)~\cite{shn89} one part of the hypertext consists of a
hyper-novel, which can only be accessed through an article that
introduces the novel. Since the nodes in the hyper-novel can only
reach a very limited subset of the hypertext they have low status and
since they can only be reached through the hyper-novel introduction
they have a high contrastatus. A biconnected component analysis
reveals the user's position in a separate cluster.
Figure~\ref{fig:hho} shows part of the reduced graph of HHO after the
first iteration of the algorithm. For users who are browsing through
the ``Interactive fiction'' cluster, the analysis gives a crude notion
of their whereabout in hyperspace\footnote{It is interesting to
note that for some nodes in the interactive fiction cluster the
hierarchisation algorithm gave depth 9. This is an example were a
local metric (depth) suggests using aggregation rather than
hierarchisation for a crude global view.}.  We were pleased to find
that the automated metrics revealed the intended structure created by
the authors.

\begin{figure}
\vspace*{8cm}
\caption [10pt] {Part of the reduced graph of HHO. 
The location of the user is 
indicated by an arrow.}
\label{fig:hho}
\end{figure}


\subsection*{Global metrics}
\paragraph{}
We define two metrics, {\em compactness} and {\em stratum}, that indicate how
complex the hypertext is.

The compactness varies between 0 and 1, and it is 0 when the hypertext
is completely disconnected and 1 when it is completely connected.
Formally
\[
CP = \frac {Max - \sum_{i} \sum_{j} C_{ij}} {Max - Min}
\]
Where Max and Min are respectively the maximum and minimum value the converted 
distance can assume. $C_{ij}$ is the converted distance between nodes {\em i}
and {\em j}\footnote{ The maximum value of the converted distance is given by
$Max = (n^{2} - n) C $  where n is the number of nodes in the hypertext
and C is the maximum value an entry in the converted distance matrix can assume.
Usually C=K, where K is the conversion constant.
The minimum is given by $Min = (n^{2} - n) $ , since the minimum distance
between two nodes is 1.}.

Stratum is a measure that indicates if there exists a ``natural'' order 
for reading the hypertext. Maximum stratum is achieved in a linear
hypertext. If the stratum is zero the hypertext has no
hierarchy. Basically
the stratum metric indicates how much of a linear ordering there
is in the hypertext.
We define the {\em linear absolute prestige} (LAP) of a hypertext  with $n$
nodes to be the absolute prestige of a linear hypertext with $ n$ 
nodes. The stratum is define as the absolute prestige of the hypertext
divided by the linear absolute prestige. For example, in figure
the absolute prestige. The linear absolute
prestige is given by (odd number of nodes, see~\cite{bot90}):
$n^{3} - n / 4$ , that is 30, and so the stratum is $ 14/30 = 0.47$.

Note that stratum and the compactness are not independent
measures. for example, if the compactness is equal to 1 (the
hypertext is totally connected) its stratum is 0.

Readers can use the above metrics to get useful information
about the nature of the hyperspace. The stratum, for example,
is a measure for the organization of the hypertext. In~\cite{bot90},
three hypertext databases were checked: CMSC, HHO and GOVA and
their stratum compared. Though HHO has more nodes than GOVA its
stratum is five times as large (0.054 to 0.011). This reflects the 
fact that HHO is
a well structured hypertext, where GOVA is not. Similar information
can be derived from compactness measurements. These global metrics
are useful when the hyperspace is broken into biconnected components
(see previous section). In that case we can get information on the nature
of the different clusters.

\subsection*{Looking around using tree maps}
\paragraph{}
Being able to impose a hierarchical order on the hypertext graph
creates the option to use it for presentation. 
Limited display space makes the use of traditional node and link
diagrams impractical for large trees.
A novel method for the visualization
of hierarchically structured information called a tree-map is 
presented in~\cite{shne90,bb}. The tree-map visualization
method maps hierarchical information to a rectangular 2-D display
in a space-filling manner; 100 \% of the available display window space
is utilized.
Tree-maps partition the
display space into a collection of rectangular bounding boxes
representing the tree structure. The drawing of nodes within their bounding
boxes is entirely dependent on the content of the nodes.
Since the display size is fixed,
the drawing size of each node varies inversely with the size of the tree.
The efficient use of space allows large hierarchies to be displayed.

In~\cite{bb} the weight assigned to each node is used to determine 
its bounding box. The weight may represent a single domain property,
or a combination of properties. Some optional properties that can be used
for weight representation are size, the number of
siblings, ROC,  or prestige. We can display the whole
hypertext to the user, or only portion of it (current position and
down, or some pre-defined environment). In figure~\ref{fig:treemap} 
a tree-map of 
CMSC hypertext is presented.
As a weight measure the total number of siblings from a
node down (or the
size of the tree below the node) was picked.
The user's location can be displayed by a directing arrow.
Users can get the general position in the hypertext and some
notion on how they can proceed. The information is more 
meaningful after index and reference nodes are removed.
\begin{figure}
\vspace*{9.5cm}
\caption [10pt] { A tree-map of the CMSC hypertext after the index and 
reference nodes were removed. The black rectangle presents the
position of the user (``research in computer systems''). The node
on the right side, that has 31 children, ``undergraduate courses''. }
\label{fig:treemap}
\end{figure}


\subsection*{Using the structure when looking around - fisheye views}
\paragraph{}
As was mentioned before, as the hypertext becomes larger, it is impractical
to display the entire tree. A possible solution is to use fisheye views.
Furnas~\cite{fur86} based his model on the observation that humans usually
represent their own neighborhood with great detail, yet only major landmarks
further away. Furnas suggests a ``degree of interest'' (DOI) function 
which assigns
a value to each node in accordance to the degree with which the user is 
interested in seeing that node. The DOI function for node $ x$ given that
the reader is in node $ y$ is defined as
\[ DOI_{fisheye} (x|y) = API(x) - D(x,y)
\]
where $ API(x)$ is the global a priori importance of $ x$ and $ D(x,y)$
is the distance between $ x$ and $ y$. The fisheye view is created 
after picking a threshold, and displaying only nodes with $DOI$ greater
than the threshold.

Several options exist for choosing our $ API$. A natural selection
might be the distance from the root. This choice gives (according to
the threshold value) the main parts of the tree. Another option can 
be to take the ROC metric and use it as $ API$. As we mentioned 
before we limited  our analysis to structural analysis only.
Better results can be achieved
by combining the structural analysis with a stylistic and context
analysis. This is another natural place to use semantic information
in determining the $ API$. Another option is to use the user behavior - like 
the number of times a node was visited. 

In figure~\ref{fig:fisheye} we 
present two fisheye views, one with the user in the 
node ``Shneiderman, Ben'' in
the CMSC hypertext, and the other being at the node ``Hyperties''.
The views were created 
using the prestige metric as the $ API$
function. Note the difference between the two views,
although the points of view are from neighboring nodes
. The user can change the threshold value to get
a more/less detailed view. Another option is to change the negative
weight that is given to the distance. Doubling the punishment on
being far will result in showing only nodes with very high $ API$
values, and the immediate neighborhood. 

\begin{figure}
\vspace*{8.75cm}
\caption [10pt] {A fisheye view of the CMSC hypertext being in 
nodes ``Shneiderman, Ben'' and ``Hyperties''.}
\label{fig:fisheye}
\end{figure}

\section{How do I get to any destination?}
\paragraph{} 
Knowing where we are is a first step toward a natural goal - getting
there. Getting there at once (jumping via an index to the target node)
is not a hard to achieve goal. Most hypertext systems give the option
to move directly to a certain node (via search, or an index). Assuming
that the user has some purpose in mind while wandering about in the
hyperspace we see the path that the user takes toward the goal to be
as important as getting there.  *** might put here the learning from
the paper *** A structural analysis of the hypertext can suggest some
alternatives in planning this path.\footnote{This direction suits
nicely in the learning support environment~\cite{ham} under
orienteering as well as rambling. Hammond and Allinson present the
need for tools which ``not only support passive navigation around the
information but also permit learners to structure the information and
the learning activities in accordance with their requirements.''  On
top of the structure imposed on the hypertyext the system can offer
the novices a meaningful (from structural point of view) path between
two nodes.}

Our toolbox gives users the option to impose hierarchical structure
on the hypertext. Users can pick the target node on the hierarchy
and get two kinds of paths: the shortest (cross reference links are 
allowed), or ``follow the tree'' path (shortest on the tree, no cross
reference links are allowed). (Do we need an Example here?).
Being able to present semantic clusters, the toolbox gives users
the option to plan a path that will bring structure into consideration
(for example bypass them). Figure~\ref{fig:hho} shows part of the reduced 
graph of the HHO hypertext, when users are somewhere in the ``Interactive
fiction'' cluster. For users who consider to go directly and
browse through the ``Travel guide''  section, the reduced graph can 
help in making that decision. The separation might suggest that there
is not much ``to lose'' by bypassing the core cluster.
 
Another option that users have is to plan a path 
through landmarks. The landmarks can be viewed using a fisheye
presentation. Another option is to pick the landmarks by
checking some structural metrics
like prestige, and plan the path through these nodes.
To carry out this option the system has to be able to suggest a
path to users that is constrained by the landmarks the users
have picked (This can be done by a collection of local shortest
paths). 

Structural analysis is only one dimension to consider while
planning a path. Adding constraints from other dimensions
(content, stylistic, etc) makes the mission of helping
users to plan a path a complex job. This part is left as
a direction for future research.

\section{Conclusions}
\paragraph{}
Analyzing the structure of a hypertext database can give useful
information to the traveler in hyperspace. We presented
a preliminary collection of structural tools (processes and metrics) for users
of hypertext systems. These tools can suggest answers to
questions like: Where am I? How do I get to any destination?
What can be seen from here looking around? etc. 
The crude positioning in the hyperspace was based on
two processes: hierarchisation and cluster identification.
Several metrics were presented and used in the above processes
as well as for locating landmarks and getting global information
on the hypertext structure (compactness, stratum, etc).
It should be emphasized that the analysis, and the
answers we get, reflect the input that we are using - the structure.
We can not get more than what a structural analysis can give. 
The results can be much stronger if the analysis were
based on the textual and the stylistic dimensions as well. This 
extension as well as the multi-constraints path planning is a direction
for future research.


\begin{thebibliography}{99}

\bibitem{kms}
  Akscyn, R. M., McCracken, D. L. \& Yoder, E. A. (1988).
  KMS: a distributed hypermedia system for managing knowledge in
  organization.
  {\em Communications of the ACM}, 31 (7), 820-835.

\bibitem{bot90}
  Botafogo, R. A., Rivlin, E., \& Shneiderman, B. Structural
  analysis of hypertext: Identifying hierarchies and useful metrics.
  {\em ACM Trans. on Information Systems}. (to appear).

\bibitem{bot91}
  Botafogo, R. A., \& Shneiderman, B. (1991).
  Identifying aggregates in hypertext structures.
  {\em Proceedings of the Hypertext 91 Conference}, 
  Available from ACM, New York, NY.

\bibitem{cha87}
  Charney, D. (1987). Comprehending non-linear text: The role of discourse
  cues and reading strategies.
  {\em Proceeding of the Hypertext 87 Conference}
  (pp. 109-120),
  Chapel Hill, North Carolina.

\bibitem{fur86}
  Furnas, G. W. (1986). Generalized fisheye views.
  {\em Proceeding of the CHI 86 Conference},
  (pp. 16-23). Available from ACM, New York, NY.

\bibitem{hal88}
  Halasz, F. G. (1988). Reflections on NoteCards: Seven issues
  for the next generation of hypermedia systems. 
  {\em Communications of the ACM}, 31 (7), 836-852.

\bibitem{ham}
  Hammond, N. \& Allinson, L. (1988).
  Travels around a learning support environment: rambling,
  orienteering or touring?
  {\em Proceeding of the CHI 88 Conference},
  (pp. 269-273). Available from ACM, New York, NY.

\bibitem{har59}
  Harary, F. (1959). Status and contrastatus. {\em Sociometry},
  22, 23-43.

\bibitem{bb}
  Johnson, B. \& Shneiderman, B. (1991).
  Tree-Maps: A space-filling approach to the visualization of 
  hierarchical information structures.
  {\em Proc. IEEE Visualization 91 Conference}.

\bibitem{mar89}
  Marshall, C. C. (1989). Guided tours and on-line presentations: How
  authors make existing hypertext intelligible for readers.
  {\em Proceedings of the Hypertext 89 Conference}, 15-26. Pittsburgh,
  Pennsylvania.

\bibitem{nie90}
  Nielsen, J. (1990). The art of navigating through hypertext.
  {\em Communications of the ACM}, 33 (3), 296-310.

\bibitem{nor}
  Norman, K. L. (1991). Personal communication. 

\bibitem{pla}
  Plaisant, C. (1991).
  Guide to opportunities in volunteer archaeology. Case study
  on the use of hypertext system in a museum exhibit.
  in {\em Hypertext/Hypermedia Handbook}
  Berk, E. \& Devlin, J. editors.
  San Diego, CA. McGraw-Hill Publishing.

\bibitem{shn89}
  Shneiderman, B., \& Kearsley, G. (1989).
  {\em Hypertext Hands-On!}. Reeding, MA. Addison-Wesley Publishing.

\bibitem{shne90}
  Shneiderman, B. (1992).
  Tree visualization with Tree-Maps: A 2-d space-filling approach.
  {\em ACM Transactions on Graphics}, January 1992.

\bibitem{van83}
  van Dijk, T., \& Kintsch, W. (1983). {\em Strategies of discourse 
  comprehension.}
  New York: Academic Press.



\end{thebibliography}
\end{document}



